%---------------------------------------------------------------------------------------------------------------------------------------------
% PAG 41
\newpage
\fancyhf{}
\fancyhead[r]{\thepage}
\begin{center}
{\fontsize{16}{18}\selectfont \textbf{SISTEMA}}
\end{center}
\vspace{0.5cm}

{\fontsize{13}{15}\selectfont
... sido blandida como una hacha ideológica destinada a suprimir los derechos de individuos y grupos en nombre del bien del todo o sistema superior, 
en particular el establishment económico-político, independientemente de si este último trabaja de hecho para el bien común. 
No perderemos tiempo en esta hoja de parra.

El holismo, en pocas palabras, es antianalítico y por tanto anticientífico. De hecho, ha sido responsable del atraso de las ciencias no físicas. 
Y ha contribuido muy poco a la sistémica seria precisamente porque (a) no se ha comprometido en un estudio de los vínculos que mantienen unido a cualquier sistema, 
y (b) en lugar de construir sistemas conceptuales (teorías) para dar cuenta de sistemas concretos, se ha desgastado atacando el enfoque analítico o atomístico y elogiando la totalidad como tal. 
Cualquier verdad que haya en el holismo -a saber, que hay totalidades, que tienen propiedades propias y que deben tratarse como totalidades- está contenida en el sistemismo, o la filosofía que sustenta la sistémica o la teoría general de sistemas (cf. Bunge, 1977d).

Al oponernos al holismo no adoptamos su opuesto, a saber, el atomismo -la tesis de que el todo está de alguna manera contenido en sus partes, de modo que el estudio de estas últimas debería bastar para comprender el primero. 
Ciertos todos, a saber, los sistemas, tienen propiedades colectivas o sistémicas que no poseen sus componentes, y es por eso que deben estudiarse como sistemas. Considere el célebre aunque poco comprendido ejemplo de una llamada identidad contingente, 
a saber, Agua = \( H_2O \). Esto no es una identidad en absoluto porque, mientras que el LHS es la abreviatura de 'cuerpo de agua' (por ejemplo, un lago), el RHS describe una propiedad de sus componentes moleculares. (No es posible la identidad entre una cosa y una propiedad.) 
Lo que es cierto, por supuesto, es que la composición molecular del agua es un conjunto de moléculas de \( H_2O \), pero esto no es una declaración de identidad. (En otras palabras, la declaración correcta es esta: Para cualquier cuerpo de agua \( w, \mathcal{C}(w) \subset \) El conjunto de moléculas de \( H_2O \).) 
Además, especificar la composición de un sistema no basta para caracterizarlo como un sistema: debemos agregar una descripción de la estructura del sistema. Y resulta que el agua, como un sistema compuesto de miríadas de moléculas de \( H_2O \), tiene propiedades que ninguno de sus componentes tiene -por ejemplo, 
transparencia, un alto poder dieléctrico (de ahí un alto poder de disolución), se congela a \( 0^\circ C \), y así sucesivamente. Algunas de estas propiedades deben incluirse en cualquier modelo realista del agua.

Las diferencias ontológicas entre un cuerpo de agua y una molécula de \( H_2O \) son tales que, 
para dar cuenta del comportamiento del primero, necesitamos no solo todo el conocimiento que tenemos sobre la molécula individual de \( H_2O \)
}

%---------------------------------------------------------------------------------------------------------------------------------------------
% PAG 42
\newpage
\fancyhf{}
\fancyhead[l]{\thepage}
\begin{center}
{\fontsize{16}{18}\selectfont \textbf{CAPÍTULO 1}}
\end{center}
\vspace{0.5cm}

{\fontsize{13}{15}\selectfont
sino también una gran cantidad de hipótesis y datos concernientes a la estructura del agua (es decir, la configuración relativa de las moléculas de \( H_2O \) en la red) 
así como hipótesis y datos sobre la dinámica de los cuerpos de agua – hipótesis y datos que varían, por supuesto, según si el agua está en fase gaseosa, líquida o sólida. 
En resumen, para describir, explicar o predecir las propiedades del agua usamos tanto microleyes como macroleyes.

El atomismo, una doctrina ontológica, suele estar, aunque no necesariamente, aliado con el reduccionismo, la doctrina epistemológica según la cual el estudio de un sistema es reducible al estudio de sus componentes. 
(Lo contrario es falso: uno puede ser un reduccionista epistemológico, pero reconocer totalidades, emergencia y niveles.) El reduccionista afirmará, por supuesto, que podemos usar macroleyes y, 
en general, leyes de sistemas, como una conveniencia, aunque en principio deberíamos poder arreglárnoslas solo con microleyes (o leyes de componentes), ya que las primeras son reducibles a (deducibles de) las últimas. 
Esta tesis contiene un grano de verdad pero no es toda la verdad. Ninguna teoría \( T_2 \) del agua como cuerpo se sigue únicamente de una teoría microfísica \( T_1 \) de la molécula de \( H_2O \) – 
ni siquiera por nuestra adjunción de lo que algunos filósofos llaman las leyes puente que relacionan conceptos macrofísicos (por ejemplo, presión) con conceptos microfísicos (por ejemplo, impacto molecular). 
Mucho más que esto debe agregarse a la teoría primaria o reductora \( T_1 \) para obtener la teoría secundaria o reducida, a saber, hipótesis concernientes a las interacciones entre los componentes del sistema.

El caso extremo de reducción es el de la deducción directa de un conjunto dado de premisas, o reducción fuerte. Ejemplos: reducción de la mecánica de partículas a la mecánica de medios continuos, y de la óptica geométrica a la óptica ondulatoria. 
(Las reducciones inversas son imposibles.) Estos son casos bastante excepcionales. En general debemos recurrir a una estrategia más compleja, a saber, la reducción débil, o deducción de una teoría primaria en conjunción con un conjunto de conjeturas 
y datos afines pero ajenos a la primera. La estructura de esta inferencia es:

\[T_1 \cup \text{ Hipótesis y datos subsidiarios concernientes a interacciones entre componentes } \models T_2.\]

Las hipótesis subsidiarias constituyen un modelo de la composición y estructura del sistema. Dado que este modelo, aunque expresado en el lenguaje de \( T_1 \), no está incluido en \( T_1 \), 
estamos en presencia no de una reducción directa (o fuerte) sino de una reducción parcial (o débil). (Cf. Bunge, 1977f.)

Nótese que no estamos afirmando que las propiedades del agua, o de cualquier otro macrosistema, sean misteriosas. Por el contrario, pueden explicarse al
}

%---------------------------------------------------------------------------------------------------------------------------------------------
% PAG 43
\newpage
\fancyhf{}
\fancyhead[r]{\thepage}
\begin{center}
{\fontsize{16}{18}\selectfont \textbf{SISTEMA}}
\end{center}
\vspace{0.5cm}

{\fontsize{13}{15}\selectfont
menos en esbozo. Por ejemplo, el punto de ebullición excepcionalmente alto y el calor de evaporación del 
agua se explican en términos de los enlaces de hidrógeno que unen todas las moléculas de \( H_2O \) en un cuerpo de agua, 
enlaces que a su vez se explican por la composición y la estructura de la molécula de \( H_2O \). Pero el punto es que los enlaces de 
hidrógeno intermoleculares no ocurren en el estudio de la molécula individual de \( H_2O \). En otras palabras, 
aunque el agua está compuesta de moléculas de \( H_2O \) no se reduce a \( H_2O \) – a pesar de los esfuerzos de filósofos ilustrados por aplastar el monstruo holista 
(por ejemplo, Kemeny y Oppenheim, 1956; Oppenheim y Putnam, 1958, Putnam, 1969).

En resumen, el atomismo es casi tan falso como el holismo, la diferencia radica en que, 
mientras el primero estimula la investigación, el segundo la bloquea. Cada una de estas visiones tiene un grano de verdad que la visión sistémica preserva y expande.

\vspace{0.5cm}
\begin{center}
{\fontsize{15}{17}\selectfont \textbf{5. COMENTARIOS FINALES}}
\end{center}
\vspace{0.5cm}

La idea de un sistema, como distinto de un agregado suelto, es muy antigua. Sin embargo, solo recientemente ha sido elucidada y explotada sistemáticamente. 
La mera sugerencia de que la cosa que estamos mirando, manipulando o investigando podría ser un sistema en lugar de un objeto no estructurado o un mero montón, 
guiará nuestro estudio y manejo del mismo. De hecho, si sospechamos que cierta cosa es un sistema entonces nos esforzaremos por identificar su composición, su entorno y su estructura.

El orden en el que aparecen las tres coordenadas del concepto de sistema es natural más que accidental. De hecho, enumerar los componentes de un sistema debe preceder 
a cualquier pregunta respecto a su entorno y su estructura; y la identificación del entorno viene antes de la exhibición de la estructura, 
porque esta última es la colección de relaciones entre los componentes y entre estos y los elementos ambientales. Cierto, cuando nos encontramos con ciertos sistemas, 
como una planta, un reloj o una galaxia, a menudo comenzamos nuestra búsqueda con la totalidad y su entorno, terminando por descubrir su composición y su estructura. 
Pero cuando investigamos un bosque, un sistema social, y a fortiori un supersistema social como una nación, primero encontramos sus componentes (o partes atómicas) en su medio e intentamos descifrar 
la estructura del todo estudiando el comportamiento de los componentes individuales. En cualquier caso, es decir, cualquiera que sea nuestro modo de percepción, 
el análisis conceptual de un sistema debe proceder de la manera indicada – identificación de composición, entorno y estructura – aunque solo sea por razones matemáticas.
}

%---------------------------------------------------------------------------------------------------------------------------------------------
% PAG 44
\newpage
\fancyhf{}
\fancyhead[l]{\thepage}
\begin{center}
{\fontsize{16}{18}\selectfont \textbf{CAPÍTULO 1}}
\end{center}
\vspace{0.5cm}

{\fontsize{13}{15}\selectfont
De hecho, no tiene sentido hipotetizar ninguna relación sin saber cuáles pueden ser los relata 
(componentes del sistema y unidades ambientales). Por lo tanto, la afirmación holista de 
que el análisis atomístico o el método de las partes es incapaz de captar totalidades, es infundada. 
Por el contrario, el holismo es incapaz de dar cuenta de cualquier totalidad precisamente porque se niega 
a revelar los componentes que se mantienen unidos en el sistema: sin componentes, no hay vínculos entre ellos. 
Esto no es para condonar la antítesis del holismo, a saber, el atomismo, y su usual compañero epistemológico, el reduccionismo, 
según el cual los todos – en particular, los sistemas – son artefactos, los emergentes son idénticos a los resultantes, 
y los niveles son solo categorías metodológicas convenientes.

La visión del mundo que emerge de este capítulo es sistémica: sostiene que el universo es un sistema compuesto de subsistemas. 
Más precisamente, el universo es el supersistema de todos los demás sistemas. El mundo no es así ni un bloque sólido ni un montón de elementos desconectados. 
Está mantenido unido por una serie de vínculos, desde enlaces intermoleculares hasta la gravitación y la información. El mundo es material pero no solo un montón de entidades físicas: 
está compuesto de sistemas de un número de tipos cualitativamente diferentes. Aunque todos los sistemas son físicos no todos ellos son solo físicos. 
El universo es enormemente variado: sus componentes pueden agruparse en una serie de niveles, como el físico, el químico, el biológico y el social. Además, el mundo es inquieto y todos sus cambios están modelados (legítimos).
El mundo es, en resumen, un sistema coherente o integrado de sistemas, y uno que es variado, cambiante y regular.
}