% =================================================================
% CONFIGURACIÓN DE DOCUMENTO Y LENGUAJE
% =================================================================
\usepackage[spanish]{babel}
\usepackage[utf8]{inputenc}
\usepackage[T1]{fontenc} % <- AÑADE ESTA LÍNEA

\usepackage{graphicx} % Necesario para incluir imágenes
\usepackage{amsmath, mathtools, amssymb} % Paquetes matemáticos
\usepackage{amsfonts} % Necesario para los comandos \mathfrak
\usepackage{float} % Mejor control sobre la posición de las figuras
\usepackage[none]{hyphenat} % Deshabilita la separación de sílabas (opcional)

% =================================================================
% CONFIGURACIÓN DE PÁGINA Y MÁRGENES (GEOMETRY)
% =================================================================
\usepackage[papersize={216mm,271mm},tmargin=20mm,bmargin=20mm,
lmargin=20mm,rmargin=20mm]{geometry}

% =================================================================
% CONFIGURACIÓN DE BIBLIOGRAFÍA (BIBLATEX) - COMENTADA TEMPORALMENTE
% =================================================================
%\usepackage[backend=biber]{biblatex} % <- COMENTA ESTA LÍNEA
%\bibliography{referencias} % <- COMENTA ESTA LÍNEA

% =================================================================
% CONFIGURACIÓN DE DIBUJO Y DIAGRAMAS (TIKZ)
% =================================================================
\usepackage{tikz}
% Todas las librerías necesarias para tus diagramas (incluyendo las de TikZ y TikZ-Circuit)
\usetikzlibrary{mindmap}
\usetikzlibrary{shapes.geometric, positioning} 
\usetikzlibrary{shapes,arrows}
\usepackage{tikz-cd} % Para diagramas conmutativos

\usepackage{circuitikz}
\ctikzset{bipoles/thickness=1.2}

\usepackage{pgfplots}
\pgfplotsset{width=10cm,compat=newest}

% =================================================================
% CONFIGURACIÓN DE ENCABEZADOS Y PIES DE PÁGINA (FANCYHDR)
% Adaptada para poner el número de página a la derecha y el texto a la izquierda
% =================================================================
\usepackage{fancyhdr}
\pagestyle{fancy}
\fancyhf{} % <- CAMBIA ESTO: Limpia todos los encabezados y pies de página

% ENCABEZADO DERECHO: Número de página (ej. 7)
\fancyhead[R]{\thepage} 

% ENCABEZADO IZQUIERDO: Texto "SYSTEM" (o lo que desees)
\fancyhead[L]{SYSTEM} 

% Elimina la línea de encabezado (headrule) para un estilo más limpio
\renewcommand{\headrulewidth}{0pt} 

% FIN DE LA CONFIGURACIÓN